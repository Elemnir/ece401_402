\documentclass[12pt]{article}

\usepackage{enumitem, setspace}
\usepackage[margin=1in]{geometry}

\doublespacing

\author{
  Howard, Adam\\
  \texttt{ahowar31@vols.utk.edu}
  \and
  Rogers, Jeremy\\
  \texttt{jroger44@vols.utk.edu}
  \and
  Seals, Matthew\\
  \texttt{mseals1@vols.utk.edu}
  \and
  Reynolds, John\\
  \texttt{jreyno40@vols.utk.edu}
  \and
  Marx, Maurice\\
  \texttt{mmarx@vols.utk.edu}
}

\title{Analysis \& Solution Strategy Report}
\date{\today}

\setlistdepth{6}

\newlist{packed_enum}{enumerate}{6}
\setlist[packed_enum]{label*=\textbf{\arabic*.},leftmargin=*}

\begin{document}
  \maketitle
  \pagebreak
  \noindent{\textbf{\large Introduction}} 
  \vspace{16 pt}

  While adhering to the needs and requirements specification for our project, our team has fleshed out and
  developed an in-depth solution. The ``Bluetooth Locator Sticker'' does not require many components, as our
  project has little need for features. This has allowed us to keep our design simple. Furthermore, a small
  requirement for features allows us to spend less on hardware for manufacturing. To successfully create a 
  product meeting the buyer's standards, we must provide a cheap and effective locator device for use within
  close range. The need for close-range usage immediately eliminates GPS technology. We quickly realized our
  choice of wireless technology narrowed to either Bluetooth or RFID. The team ultimately decided on Bluetooth.
  Bluetooth MAC addresses add implicit uniqueness that RFID cannot provide. If RFID was used, unique
  identification would have to be added during the production process; thus, it was decided that Bluetooth
  is the better solution.

  \vspace{16 pt}

  \begin{packed_enum}
    \item \textbf{Device}
    \begin{packed_enum}
      \item \textbf{Physical Design}
      \begin{packed_enum}
        \item \textbf{Package Body} \\
        The package body should be within 1-3 grams in weight to maintain a negligible added weight to the 
        flying discs. The device housing must also be nature-resistant, i.e., it must resist small amounts of
        force, water, and other elements. The packaging must not affect the aerodynamics of the disc, and it
        must cause a minimal amount of signal attenuation. Plastic (or a similar material) is a good choice.
        \item \textbf{Package Interface} \\
        The package must provide some functionality that allows the user to directly interface with the device.
        An access port must be provided which allows the user to charge the device. This could be implemented 
        as a simple USB port. There must also be a button present on the housing that allows the user to turn
        the device on/off. The button must minimize the risk of accidental triggering from environmental
        elements. A conductive pad could be used to fulfill the button requirement. The package is required
        to contain status LEDs, which relate device status information to the user. A single LED can be used
        to indicate low battery power. A third LED can be used to indicate whether the device is currently
        paired with a mobile device. This LED will temporarily blink after successful pairing, so the user knows
        that the device was successfully connected. In lieu of a three LED system, a single RGB led using three
        different colors could be used.
        \item \textbf{Adhesive Pad} \\
        An adhesive pad is required to attach a device to a frisbee disc. The pad must be sturdy, but also
        easily removable from the device and target. A double-sided sticker could be used as the adhesive pad.
      \end{packed_enum}
      \item \textbf{Software Specifications}
      \begin{packed_enum}
        \item \textbf{Modes of Operation} \\
        The software of the device has two primary modes. One mode is active when the device is communicating 
        with the application, and the other mode is active when the device is not communicating with the
        application. The device must use information from its hardware to determine which state it is in. When
        communicating with a mobile device, the Bluetooth device should remain active, responding to input from
        the application. If the Bluetooth device is not recieving input from an application, it should be in a 
        low-power mode, waking periodically to broadcast its MAC address and listen for responses. The modes of
        operation have major implications on the battery life of the device, so it is important that they are
        implemented correctly.
        \item \textbf{State Monitoring} \\
        State Monitoring allows the device to make decisions about its modes of operation based on input from
        peripherals. The device is required to estimate its total remaining battery life, and reduce the
        frequency of wake cycles when in power-saving mode. This requirement improves the battery life of the
        device.
      \end{packed_enum}
      \item \textbf{Hardware Specifications}
      \begin{packed_enum}
        \item \textbf{Bluetooth Module} \\
        The bluetooth module is the most important component composing the device. It must be self-contained,
        should not require a separate antenna, have an effective range of 50 meters, and be able to wake and 
        sleep periodically. The RN4020 Bluetooth module fulfills the requirements, and was chosen as the module
        to use in the device.
        \item \textbf{Speaker} \\
        The speaker is required to notify the user of the device's location. It must accept input from the
        Bluetooth module, and produce a tone audible to the user from around 10 meters away. A piezoelectric
        buzzer produces the required tone, and can communicate with the bluetooth module.
        \item \textbf{Battery} \\
        A battery is necessary to power the components of the device. It must be rechargeable and last for
        several hours. A lithium-ion battery fulfills the requirements.
      \end{packed_enum}
      \item \textbf{Tolerances} \\
      Since the device is used outdoors, it must be resilient to environmental factors. The device must be
      operable in temperature ranges that are common in moderate climates, as well as be water-resistant. The
      requirement of water-resistance affects the type of adhesive pad that can be used, and requires the
      housing of the device to be resilient. The components discussed thus far are suitable for use considering
      these factors.
    \end{packed_enum}
    \item \textbf{Application}
    \begin{packed_enum}
      \item \textbf{User Interface} \\
      The mobile application has two main functions that are incorporated into the design of the user interface.
      Firstly, it needs to allow a user to pair a Bluetooth sticker with the mobile device, and secondly, it 
      needs to allow the user to view the location of any paired Bluetooth sticker. The usage patterns of the
      two main functions impact the design of the user interface. Pairing a sticker with a device is a task that
      is not done frequently by the user. The user will, however, very often need to access the location of a
      certain sticker. Two different views are provided by the user interface, one for each of the two functions
      described above.
      \begin{packed_enum}
        \item \textbf{Device Registration View} \\
        The device registration view provides an interface that shows all Bluetooth stickers currently paired
        with the mobile device, as well as all unpaired Bluetooth stickers in range that could potentially be
        paired. Since there are essentially two classes of Bluetooth stickers, paired and unpaired, the device
        registration view needs to make the distinction between the two. A possible solution is to have the view
        provide two lists, one that has all currently paired devices, and one which shows all devices in range.
        The drawback of this solution is that it requires an automatic scan of Bluetooth devices when the user
        accesses the device registration view. If a user simply wants to view the devices registered, battery is
        wasted in scanning for other devices. An alternative is to have the view initially provide a single list
        containing all currently paired devices. If the user wants to scan for devices, he or she must
        explicitly do so, possibly by having a dedicated button on the view. Clicking the button will provide a
        list of all potential devices, and the user can select a device to pair. \\
        The user needs to be able to customize the identifier for any given device. Consequently, the view needs
        to provide the ability to edit the name associated with any given device. This functionality can be
        achieved by having a simple 'edit' option for each device.
        \item \textbf{Device Location View} \\
        The device location view should allow a user to view the signal strength of a specific device. The
        signal strength is implemented as a simple meter, which visually notifies the user of the distance of
        the device from his or her current location. The user needs to be able to activate the buzzer of a
        device, and this functionality is tied into the device location view. When a specific device is in
        range, a button appears in the view that activates the buzzer of the device currently being located.
      \end{packed_enum}
      \item \textbf{Device Communications}
      \begin{packed_enum}
        \item \textbf{One-way Communications} \\
        The device will periodically send out a simple `ping' with its MAC address, and wait for any connections
        to be attempted on it. 
        \item \textbf{Two-way Communications} \\
        Once the device receives a signal back to it, i.e., the mobile device attached to its MAC, it will 
        continue to send out pings to determine the range. This will be done via RSSI in Bluetooth, which will
        determine the signal strength, and thus, the relative distance of the device away from the mobile
        device. 
        \item \textbf{Out of Range} \\
        When the device is out of range, it will simply continue to do the one-way communications method until the battery dies. This way, it will
        be able to be found regardless of how long ago it was lost, assuming that the battery is still alive.
      \end{packed_enum}
      \item \textbf{Platform Compatibility} \\
      The application is required to be available for any Bluetooth-enabled Android or iOS device. This
      requirement essentially requires two different mobile applications; one programmed using Java for Android
      with the Android Bluetooth API, and another programmed in Objective-C (or Swift) for iOS using the iOS
      Bluetooth API. The application is made available to the public by publishing it in the Google Play Store
      and the App Store.
    \end{packed_enum}
  \end{packed_enum}

  
\end{document}
